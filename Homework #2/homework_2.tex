\documentclass[12pt]{article}
 
\usepackage[margin=1in]{geometry}
\usepackage{amsmath,amsthm,amssymb, mathtools}
\usepackage[T1]{fontenc}
\usepackage{lmodern}
\usepackage{fixltx2e}
\usepackage[shortlabels]{enumitem}
\usepackage{mathrsfs}
\usepackage{kbordermatrix}

\usepackage{graphicx}

\renewcommand{\kbldelim}{(}% Left delimiter
\renewcommand{\kbrdelim}{)}% Right delimiter
 
\newcommand{\N}{\mathbb{N}}
\newcommand{\R}{\mathbb{R}}
\newcommand{\Z}{\mathbb{Z}}
\newcommand{\Q}{\mathbb{Q}}
 
\newenvironment{theorem}[2][Theorem]{\begin{trivlist}
\item[\hskip \labelsep {\bfseries #1}\hskip \labelsep {\bfseries #2.}]}{\end{trivlist}}
\newenvironment{lemma}[2][Lemma]{\begin{trivlist}
\item[\hskip \labelsep {\bfseries #1}\hskip \labelsep {\bfseries #2.}]}{\end{trivlist}}
\newenvironment{exercise}[2][Exercise]{\begin{trivlist}
\item[\hskip \labelsep {\bfseries #1}\hskip \labelsep {\bfseries #2.}]}{\end{trivlist}}
\newenvironment{problem}[2][Problem]{\begin{trivlist}
\item[\hskip \labelsep {\bfseries #1}\hskip \labelsep {\bfseries #2.}]}{\end{trivlist}}
\newenvironment{question}[2][Question]{\begin{trivlist}
\item[\hskip \labelsep {\bfseries #1}\hskip \labelsep {\bfseries #2.}]}{\end{trivlist}}
\newenvironment{corollary}[2][Corollary]{\begin{trivlist}
\item[\hskip \labelsep {\bfseries #1}\hskip \labelsep {\bfseries #2.}]}{\end{trivlist}}
\newcommand{\textfrac}[2]{\dfrac{\text{#1}}{\text{#2}}}
\newcommand{\floor}[1]{\left\lfloor #1 \right\rfloor}

\newenvironment{amatrix}[1]{%
  \left(\begin{array}{@{}*{#1}{c}|c@{}}
}{%
  \end{array}\right)
}

\DeclareMathOperator*{\E}{\mathbb{E}}


\begin{document}

\title{Dynamical Systems: Homework 2}

\author{Chris Hayduk}
\date{September 29, 2020}

\maketitle

\begin{problem}{1}
\end{problem}

We have,
\begin{align*}
\Sigma = \Sigma_b = \{X = (x_k)_{k = -\infty}^{\infty}: x_k \in \{0, \cdots, b - 1\}\}
\end{align*}

and,
\begin{align*}
d(x,y) = d_{\theta}(x,y) = \begin{cases*}
      0 & \text{if} \; x = y \\
      \theta^{\min\{|k|: x_k \neq y_k\}} &
    \end{cases*}
\end{align*}

where $0 < \theta < 1$.\\

Need to show that $(\Sigma, d)$ is compact.\\

Let us suppose that $(\Sigma, d)$ is closed and bounded but not compact, and then prove that this gives us compactness.\\

Consider $C_{-k, k}(x) = \{y \in \Sigma: y_{-k} = x_{-k}, \cdots, y_k = x_k\}$. Then
\begin{align*}
C_{-k, k}(x) &= B(x, \theta^k)\\
&= \{y \in \Sigma: d_{\theta}(x,y) < \theta^k\}
\end{align*}

Hence, $C_{-k, k}(x)$ is open. $C_{-k, k}$ is also closed.\\

$C_{-1, 1} = \{x_{-1}, x_0, x_1\}$, $x \in \Sigma$. There are $b^3$ possibilities.\\

For $C_{-1, 1}$, we have
\begin{align*}
C_{-1, 1}(x) = \cup_{y \in C_{-1, 1}(x)} C_{-2, 2}(y) 
\end{align*}

So there are $b^2$ cylinders.\\

Since $\Sigma$ is assumed to be bounded, we have a sequence of cylinders $C_{\ell}$ in $\Sigma$ such that $\Sigma \cap C_{\ell}$ does not have a finite subcover.\\

We claim $(\Sigma \cap C_{\ell}) \to 0$.\\

For every $\ell$, there exists $x_{\ell} \in \Sigma \cap C_{\ell}$\\

This implies that $(x_{\ell})_{\ell}$ is a Cauchy sequence. Since $\Sigma$ is complete, the limit of this sequence exists and we write,
\begin{align*}
x = \lim_{\ell \to \infty} x_{\ell}
\end{align*}

Since $\Sigma$ is assumed to be closed, we have $x \in \Sigma$\\

Hence, $\exists \lambda_x$ such that $x \in U_{\lambda_x}$ with $U_{\lambda_x}$ open.\\

Thus, $\exists r > 0$ such that $B(x, r) \subset U_{\lambda_x}$\\

Since claim $(C_{\ell}) \to 0$ and $x_{\ell} \in C_{\ell}$, we can conclude that,
\begin{align*}
C_{\ell} \subset B(x, r) \subset U_{\lambda_x}
\end{align*}

Hence, $\Sigma \cap C_{\ell} \subset U_{\lambda_x}$ and thus $\{U_{\lambda_x}\}$ is a finite subcover of $\Sigma \cap C_{\ell}$, a contradiction. Thus, $(\Sigma, d)$ is compact.

\begin{problem}{2}
\end{problem}

Suppose $\Sigma_A$ is not closed. That is, there exists a sequence $(y)_{\ell} \in \Sigma_A$ which converges to a limit not in $\Sigma_A$. Let us call this limit $y$.\\

Since $y \not\in \Sigma_A$, there exists $j \in \mathbb{Z}$ such that $a_{y_j, y_{j+1}} = 0$. Let this be the index with smallest absolute value $|j|$ such that this occurs. Since every element of the sequence $(y_k)_{\ell} \in \Sigma_A$, we have that $a_{y_{m_j}, y_{m_{j+1}}} = 1$ for every $m \in \mathbb{N}$\\

Note then that $d(y, y_m) \geq \theta^j$ for every $m \in \mathbb{N}$. Hence, if we fix $\epsilon_j < \theta^j$, we have that,
\begin{align*}
d(y, y_m) \geq \theta^j > \epsilon_j
\end{align*}

for every $m \in \mathbb{N}$. As a result, we have that $(y)_{\ell}$ does not converge, a contradiction. Hence, $\Sigma_A$ must be closed.\\

Now consider the shift map $\sigma$ (which shifts each entry in a sequence one space to the left) and the inverse shift map $\sigma^{-1}$ (which shifts each entry in a sequence one space to the right).\\

Suppose $x \in \Sigma_A$. Then $a_{x_{i}, x_{i+1}} = 1$ for every $i \in \mathbb{Z}$.
\newpage
\begin{problem}{3}
\end{problem}

We have $f_{\mu}(x) = \mu x(1 - x)$
\begin{enumerate}[label=(\alph*)]
\item Suppose $\mu = 1$. Then we have,
\begin{align*}
f_{\mu}(x) = x(1 - x)
\end{align*}

Suppose $x \in (0, 1]$. Then we have,
\begin{align*}
x(1-x) < x
\end{align*}

since $0 \leq 1 - x < 1$. Hence, if we take $x_1 = f_{\mu}(x)$, then
\begin{align*}
f_{\mu}(x_1) = f^2_{\mu}(x) = x_1(1 - x_1) <  x_1 = f_{\mu}(x) < x
\end{align*} 

This extends for all $n \in \mathbb{N}$. Note also that,
\begin{align*}
f_{\mu}(0) = 0(1 - 0) = 0
\end{align*}

and that $0 \leq f_{\mu}(x) \leq 0.25$ for every $x \in (0, 1]$. That is, if our initial $x \in (0, 1]$, then $f^n_{\mu} \in [0, 0.25]$ for any choice of $n$.\\

Thus, we have that $f^n_{\mu}(x)$ is a monotonically decreasing sequence of real numbers with lower bound $0$ if the initial $x \in (0, 1]$.\\

Now note that we have,
\begin{align*}
&x = f_{\mu}(x)\\
\iff &x = x(1-x)\\
\iff &1 = 1-x\\
\iff &x = 0
\end{align*}

so $0$ is the only fixed point of $f_{\mu}(x)$.\\

Hence, we cannot have $\lim_{n \to \infty} f^n_{\mu} > 0$ since there are no positive fixed points. In addition, we cannot have $\lim_{n \to \infty} f^n_{\mu} < 0$ since $f^n_{\mu} \geq 0$ for every $n$. Thus, we must have,
\begin{align*}
\lim_{n \to \infty} f^n_{\mu} = 0
\end{align*}

for every $x \in (0, 1]$.

\item Let $1 < \mu < 3$ and consider $p_{\mu} = \frac{\mu-1}{\mu}$. We have,
\begin{align*}
f_{\mu}\left(\frac{\mu-1}{\mu}\right) &= \mu \left(\frac{\mu-1}{\mu}\right) \left(1 - \frac{\mu-1}{\mu}\right)\\
&= (\mu - 1) \left(1 - \frac{\mu-1}{\mu}\right)\\
&= \mu - 1 - \frac{(\mu - 1)(\mu - 1)}{\mu}\\
&= \mu - 1 - \frac{\mu^2 - 2\mu + 1}{\mu}\\
&= \frac{\mu^2 - \mu - \mu^2 + 2\mu - 1}{\mu}\\
&= \frac{\mu - 1}{\mu}
\end{align*}

So we have shown that $p_{\mu} = \frac{\mu-1}{\mu}$ is a fixed point of $f_{\mu}(x)$.\\

Now consider,
\begin{align*}
&f_{\mu}(x) = \mu x(1-x)\\
\implies &f_{\mu}'(x) = \mu (1 - 2x)
\end{align*}

So,
\begin{align*}
f_{\mu}'\left(\frac{\mu-1}{\mu}\right) &= \mu \left(1 - 2\left(\frac{\mu-1}{\mu}\right)\right)\\
&= \mu - 2\mu \left(\frac{\mu-1}{\mu}\right)\\
&= \mu - 2\mu + 2\\
&= 2 - \mu
\end{align*}

Since $1 < \mu < 3$, we have,
\begin{align*}
-1 < 2 - \mu < 1
\end{align*}

and,
\begin{align*}
\left|f_{\mu}'\left(\frac{\mu-1}{\mu}\right)\right| < 1
\end{align*}

Thus, $p_{\mu}$ is an attracting fixed point.\\

Now we need to show that $W^s_{\text{loc}}(p_{\mu}) = (0, 1)$. Note that,
\begin{align*}
W^s(a) &= \{x \in X: f^n(x) \to a \; \text{as} \; n \to \infty\}\\
&= \{x \in X: \lim_{n \to \infty} |f^n(a) - f^n(x)| = 0\}
\end{align*}

and
\begin{align*}
W^s_{\text{loc}}(a) &= \text{the connected component of} \; W^s(a) \; \text{containing} \; a\\
&= \cup_{n \in \mathbb{N}_0} f^{-n} \left(W^s_{\text{loc}}(a)\right)
\end{align*}
\end{enumerate}

\begin{problem}{4}
\end{problem}

Let $f: \mathbb{R} \to \mathbb{R}$ be defined by $f(x) = \frac{1}{2}(x^3 + x)$. We have,
\begin{align*}
&x = \frac{1}{2}(x^3 + x)\\
\iff &\frac{x^3}{2} + \frac{x}{2} - x = 0\\
\iff &\frac{x^3}{2} - \frac{x}{2} = 0\\
\iff &x^3 - x = 0\\
\iff &x(x^2 - 1) = 0\\
\iff &x = 0 \; \text{or} \; x^2 = 1
\end{align*}

So, we have that the fixed points of $f(x)$ are,
\begin{align*}
f(0) = 0, f(1) = 1, f(-1) = -1
\end{align*}

Now observe that,
\begin{align*}
f'(x) = \frac{1}{2}(3x^2 + 1)
\end{align*}

so we have,
\begin{align*}
f'(0) &= \frac{1}{2}\\
f'(1) &= 2\\
f'(-1) &= 2
\end{align*}

Hence, $0$ is an attracting fixed point and $1, -1$ are repelling fixed points.\\

Observe that, if $x > 1$, then $x^3 - x > 2x$ and we get,
\begin{align*}
\frac{1}{2} (x^3 + x) > \frac{1}{2} (2x) = x
\end{align*}

Hence, for every $x > 1$, we have,
\begin{align*}
f(x) > x
\end{align*}

Now fix $x < -1$. Then we have $x^3 - x < 2x$ and we get
\begin{align*}
f(x) \frac{1}{2} (x^3 + x) < \frac{1}{2} (2x) = x
\end{align*}

for every $x < -1$.\\

Now let $x \in (-1, 0)$. We have, $x^3 - x > 2x$, so 
\begin{align*}
f(x) = \frac{1}{2} (x^3 + x) < \frac{1}{2} (2x) = x
\end{align*}

for every $x \in (-1, 0)$. Since $0$ is an attracting fixed point and $f^n(x)$ is a monotonically increasing sequence with supremum $0$ when $x \in (-1, 0)$, we have that $\lim_{n \to \infty} f^n(x) = 0$ in this case.\\

Now assume $x \in (0, 1)$. Then we have $x^3 - x < 2x$ and we get 
\begin{align*}
f(x) \frac{1}{2} (x^3 + x) < \frac{1}{2} (2x) = x
\end{align*}

for every $x \in (0, 1)$. Since $0$ is an attracting fixed point and $f^n(x)$ is a monotonically decreasing sequence with infimum $0$ when $x \in (1, 0)$, we have that $\lim_{n \to \infty} f^n(x) = 0$ in this case.\\

Thus, we get the following limit,
\begin{align*}
\lim_{n \to \infty} f^n(x) = \begin{cases} 
      -\infty & x < -1 \\
      -1 & x = -1\\
      0 & -1 < x < 1 \\
      1 & x = 1\\
      \infty & x > 1 
   \end{cases}
\end{align*}
\end{document}