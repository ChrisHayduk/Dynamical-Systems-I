\documentclass[12pt]{article}
 
\usepackage[margin=1in]{geometry}
\usepackage{amsmath,amsthm,amssymb, mathtools}
\usepackage[T1]{fontenc}
\usepackage{lmodern}
\usepackage{fixltx2e}
\usepackage[shortlabels]{enumitem}
\usepackage{mathrsfs}
\usepackage{kbordermatrix}

\usepackage{graphicx}

\renewcommand{\kbldelim}{(}% Left delimiter
\renewcommand{\kbrdelim}{)}% Right delimiter
 
\newcommand{\N}{\mathbb{N}}
\newcommand{\R}{\mathbb{R}}
\newcommand{\Z}{\mathbb{Z}}
\newcommand{\Q}{\mathbb{Q}}
 
\newenvironment{theorem}[2][Theorem]{\begin{trivlist}
\item[\hskip \labelsep {\bfseries #1}\hskip \labelsep {\bfseries #2.}]}{\end{trivlist}}
\newenvironment{lemma}[2][Lemma]{\begin{trivlist}
\item[\hskip \labelsep {\bfseries #1}\hskip \labelsep {\bfseries #2.}]}{\end{trivlist}}
\newenvironment{exercise}[2][Exercise]{\begin{trivlist}
\item[\hskip \labelsep {\bfseries #1}\hskip \labelsep {\bfseries #2.}]}{\end{trivlist}}
\newenvironment{problem}[2][Problem]{\begin{trivlist}
\item[\hskip \labelsep {\bfseries #1}\hskip \labelsep {\bfseries #2.}]}{\end{trivlist}}
\newenvironment{question}[2][Question]{\begin{trivlist}
\item[\hskip \labelsep {\bfseries #1}\hskip \labelsep {\bfseries #2.}]}{\end{trivlist}}
\newenvironment{corollary}[2][Corollary]{\begin{trivlist}
\item[\hskip \labelsep {\bfseries #1}\hskip \labelsep {\bfseries #2.}]}{\end{trivlist}}
\newcommand{\textfrac}[2]{\dfrac{\text{#1}}{\text{#2}}}
\newcommand{\floor}[1]{\left\lfloor #1 \right\rfloor}

\newenvironment{amatrix}[1]{%
  \left(\begin{array}{@{}*{#1}{c}|c@{}}
}{%
  \end{array}\right)
}

\DeclareMathOperator*{\E}{\mathbb{E}}

\newcommand{\Mod}[1]{\ (\mathrm{mod}\ #1)}

\begin{document}

\title{Dynamical Systems: Homework 4}

\author{Chris Hayduk}
\date{October 25, 2020}

\maketitle

\begin{problem}{1}
\end{problem}

Recall that the shift map $\sigma: \Sigma_A \to \Sigma_A$ is defined as follows:
\begin{align*}
\sigma((x_k)_{k=0}^{\infty}) = (x_{k+1})_{k = 0}^{\infty}
\end{align*}

That is, we are essentially cutting off the first element of the sequence.\\

In addition, recall that topological transitivity is defined as $\exists x \in \Sigma_A$ such that the orbit $O(x) = \{\sigma^n(x) : n \in \mathbb{N}_0\}$ of $x$ is dense in $\Sigma_A$.\\

Lastly, we have that,
\begin{align*}
d(x, y) = d_{\theta}(x, y) = \begin{cases}
      0 & \text{if} \; x = y \\
      \theta^{\min \{k: x_k \neq y_k\}} & \text{if} \; x \neq y
    \end{cases}
\end{align*}

where $0 < \theta < 1$.\\

Let us define $A$ as follows,
\begin{align*}
A = \kbordermatrix{
    & 0 & 1 & 2 & 3 & 4 & \cdots & d-1 \\
    0 & 0 & 1 & 1 & 1 & 1 & \cdots & 1\\
    1 & 0 & 1 & 1 & 1 & 1 & \cdots & 1\\
    2 & 0 & 1 & 1 & 1 & 1 & \cdots & 1\\
    3 & 0 & 1 & 1 & 1 & 1 & \cdots & 1\\
    4 & 0 & 1 & 1 & 1 & 1 & \cdots & 1\\
    \vdots & & & \vdots & & & \cdots &\vdots\\
    d-1 & 0 & 1 & 1 & 1 & 1 & \cdots & 1
  }
\end{align*}

That is, $0$ cannot appear anywhere in a sequence $x \in \Sigma_A$ except for in the first position. Now let us fix some $x, y \in \Sigma_A$ with $x \neq y$. Let $y$ be such that it begins with $0$. That is,
\begin{align*}
y = (0y_1y_2y_3\cdots)
\end{align*}

If $x$ does not also start with $0$, then it is clear that,
\begin{align*}
d(\sigma^n(x), y) = \theta^0 = 1
\end{align*}

for every $n \in \mathbb{N}_0$. This is true because, by the definition of $A$, the element $0$ can only appear as the first element in a sequence. Hence, if $x$ does not start with a $0$, then $O(x) \cap B(y, \epsilon) = \emptyset$ for every $y \in \Sigma_A$ that starts with $0$ and every $\epsilon < 1$.\\

Now suppose $x$ does indeed start with a $0$ and fix $\epsilon = \theta^2$. Observe that following a $0$, we have $d-1$ choices for the next element in the sequence. Let us now fix $y \in \Sigma_A$ such that $y_0 = 0$ and $y_1 \neq x_1$. Then we have,
\begin{align}
d(x, y) = \theta^1 = \theta
\end{align}

However, observe that since $0$ can only appear at the beginning of a sequence, the first element of $\sigma^n(x)$ will never be $0$ for any $n \geq 1$. Hence, we have,
\begin{align*}
d(\sigma^n(x), y) = \theta^0 = 1
\end{align*}

for every $n > 1$. Combining this result with (1) yields
\begin{align*}
d(\sigma^n(x), y) \geq \theta > \epsilon = \theta^2
\end{align*}

for all $n \in \mathbb{N}_0$. Thus, we have that $O(x) \cap B(y, \epsilon) = \emptyset$ in this case as well. Since we have covered all possible cases for $x \in \Sigma_A$, we have shown that there is no $x$ such that $O(x)$ is dense in $\Sigma_A$.


\begin{problem}{2}
\end{problem}

\begin{enumerate}[label=(\alph*)]

\item Suppose $\lambda > 1/e$. Then,
\begin{align*}
E_{\lambda}(x) &= \lambda e^x\\
&> \frac{e^x}{e}\\
&= e^{x-1}
\end{align*}

Hence, if we define $f(x) = e^{x-1}$, it suffices to show that $\lim_{n \to \infty} f^n(x) = \infty$ in order to assert that $\lim_{n \to \infty} E_{\lambda}^n(x) = \infty$.\\

From Bernoulli's Inequality, we have that,
\begin{align*}
e^x \geq x+1
\end{align*}

for all $x \in \mathbb{R}$. Hence, applying this to our case, we have that,
\begin{align*}
f(x) = e^{x-1} \geq (x-1)+1 = x
\end{align*}

Now let us define the function $y = x-1$. Then,
\begin{align*}
f(y) = e^{y} \geq y + 1
\end{align*}

Note that $y$ ranges over all of $\mathbb{R}$. Define $g(y) = y+1$. It is clear that,
\begin{align*}
g^n(y) &= y + 1 + 1 + \cdots + 1\\
&= y + n
\end{align*}

Thus,
\begin{align*}
\lim_{n \to \infty} g^n(y) &= y \lim_{n \to \infty} n\\
&= y \cdot \infty\\
&= \infty
\end{align*}

Since $E_{\lambda}(x) > f(x) = f(y) \geq g(y)$, we have that,
\begin{align*}
\lim_{n \to \infty} E_{\lambda}^n(x) > \lim_{n \to \infty} f^n(y) \geq \lim_{n \to \infty} g(y)
\end{align*}

This yields,
\begin{align*}
\lim_{n \to \infty} E_{\lambda}^n(x) > \infty
\end{align*}

and so,
\begin{align*}
\lim_{n \to \infty} E_{\lambda}^n(x) = \infty
\end{align*}

\item Now suppose $\lambda = 1/e$. Then we have,
\begin{align*}
E_{\lambda}(x) &= \lambda e^x\\
&=\frac{e^x}{e}\\
&= e^{x-1}
\end{align*}

Observe that $E_{\lambda}(1) = e^{1-1} = 1$. Hence, $1$ is a fixed point of $E_{\lambda}(x)$. In addition, we have,
\begin{align*}
E_{\lambda}'(x) = e^{x-1}
\end{align*}

So,
\begin{align*}
E_{\lambda}'(1) &= e^{1-1}\\
&= 1
\end{align*}

Hence, $1$ is a non-hyperbolic fixed point of $E_{\lambda}(x)$. Now suppose $x < 1$. Then,
\begin{align}
E_{\lambda}(x) &= e^{x-1} \nonumber \\
&< e^{1-1} = 1
\end{align}

In addition, from Bernoulli's Inequality, we have that,
\begin{align*}
e^x \geq x+1
\end{align*}

for all $x \in \mathbb{R}$. Hence, applying this to our case, we have that,
\begin{align*}
E_{\lambda}(x) = e^{x-1} \geq (x-1)+1 = x
\end{align*}

Note that since $x \neq 1$, we get the following inequality,
\begin{align}
E_{\lambda}(x) = e^{x-1} > (x-1)+1 = x
\end{align}

By combining (2) and (3) and fixing $x < 1$, we have that,
\begin{align*}
x < E_{\lambda}(x) < E_{\lambda}^2(x) < E_{\lambda}^3(x) < \cdots < 1
\end{align*}

Hence, if we define the sequence $(x_n)_{n \in \mathbb{N}_0} = E_{\lambda}^n(x)$, we have that it is monotonically increasing and bounded above by $1$. Thus, by the Monotone Convergence Theorem, $(x_n)$ converges to its supremum.\\

Now suppose that $\sup (x_n) < 1$. Then there exists $m \in \mathbb{R}$ such that $x_n \leq m < 1$ for every $n \in \mathbb{N}_0$. But observe that, since $m$ is a supremum of the set, for any $\epsilon > 0$, we have that $m < x_n + \epsilon$ for some $x_n$. If we choose $\epsilon$ small enough, we have,
\begin{align*}
x_{n+1} &= e^{x_n - 1}\\
&> x_n
\end{align*}

Hence, $x_{n+1} = x_n + \epsilon$ and so $x_{n+1} > m$, a contradiction. Thus the supremum of the set is $1$. Hence, if $x < 1$, then,
\begin{align*}
\lim_{n \to \infty} E_{\lambda}^n(x) = 1
\end{align*}

Now suppose $x > 1$. We have,
\begin{align*}
x < E_{\lambda}(x) < E_{\lambda}^2(x) < E_{\lambda}^3(x) < \cdots
\end{align*}

as before. However, we also have that this set is unbounded. Suppose that it were bounded. Then there exists a number $m > 1$ such that,
\begin{align*}
x < E_{\lambda}(x) < E_{\lambda}^2(x) < E_{\lambda}^3(x) < \cdots < m
\end{align*}

This implies that $m$ is a fixed point of $E_{\lambda}^n(x)$. However, we know that there are no fixed point of this function greater than $1$. Thus, we have a contradiction and hence there is no upper bound for this sequence. Thus, if $x > 1$, then,
\begin{align*}
\lim_{n \to \infty} E_{\lambda}^n(x) = \infty
\end{align*}


\item Suppose $0 < \lambda < 1/e$. Then
\begin{align*}
E_{\lambda}(x) &= \lambda e^x\\
&< \frac{e^x}{e}\\
&= e^{x-1}
\end{align*}

We have,
\begin{align*}
&\lambda e^x = x\\
\implies &e^x = x/\lambda\\
\implies &x = \ln(x/\lambda)\\
\implies &x = \ln(x) - \ln(\lambda)
\end{align*}
\end{enumerate}

\begin{problem}{3}
\end{problem}

Let $\alpha \in [0, 1) \cap \mathbb{Q}$. Hence, $\alpha = q/p$ for some $q, p \in \mathbb{Z}$ such that $0 \leq \alpha < 1$. Hence,
\begin{align*}
f(e^{2\pi i \phi}) &= e^{2\pi i (\alpha + \phi)}\\
&= e^{2\pi i (p/q + \phi)}
\end{align*}

Note that this map shifts the initial angle by $\alpha = p/q$. Now suppose we consider $f^q(e^{2\pi i \phi})$. That is, we iterate the function $q$ times. We are thus have that $f^q$ is equivalent to,
\begin{align*}
f^q(e^{2\pi i \phi}) &= e^{2\pi i[(\alpha + \alpha + \cdots + \alpha) + \phi]}
\end{align*} 

That is, we are shifting the initial angle by $q \cdot \alpha = p$. Hence,
\begin{align*}
f^q(e^{2\pi i \phi}) &= e^{2\pi i (q \cdot \alpha + \phi)}\\
&= e^{2\pi i (p + \phi)}\\
&= e^{2\pi i p + 2\pi i \phi}
\end{align*}

Since $p \in \mathbb{Z}$, we have that $2 \pi i p$ is a multiple of $2\pi$. Hence, $2\pi i p$ represents a full rotation around the circle, returning to the angle $\phi$. Thus,
\begin{align*}
f^q(e^{2\pi i \phi}) &= e^{2\pi i \phi}
\end{align*}

for any angle $\phi$. Hence, $\text{Per}(f) = [0, 2\pi)$.\\

Now suppose $\alpha \in [0, 1) \cap \mathbb{I}$. I would guess that $\text{Per}(f)$ would exclude any rational points, since shifting a rational number by an irrational number would result in an irrational number.

\begin{problem}{4}
Bonus problem
\end{problem}

\begin{problem}{5}
Bonus problem
\end{problem}

Let $4 < \mu \leq 2 + \sqrt{5}$ and let $\Lambda_{\mu} = \{x \in [0, 1] \ | \ f^n(x) \in [0, 1] \ \forall n \in \mathbb{N}\}$

\end{document}