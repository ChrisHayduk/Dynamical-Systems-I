\documentclass[12pt]{article}
 
\usepackage[margin=1in]{geometry}
\usepackage{amsmath,amsthm,amssymb, mathtools}
\usepackage[T1]{fontenc}
\usepackage{lmodern}
\usepackage{fixltx2e}
\usepackage[shortlabels]{enumitem}
\usepackage{mathrsfs}
\usepackage{kbordermatrix}

\usepackage{graphicx}

\renewcommand{\kbldelim}{(}% Left delimiter
\renewcommand{\kbrdelim}{)}% Right delimiter
 
\newcommand{\N}{\mathbb{N}}
\newcommand{\R}{\mathbb{R}}
\newcommand{\Z}{\mathbb{Z}}
\newcommand{\Q}{\mathbb{Q}}
 
\newenvironment{theorem}[2][Theorem]{\begin{trivlist}
\item[\hskip \labelsep {\bfseries #1}\hskip \labelsep {\bfseries #2.}]}{\end{trivlist}}
\newenvironment{lemma}[2][Lemma]{\begin{trivlist}
\item[\hskip \labelsep {\bfseries #1}\hskip \labelsep {\bfseries #2.}]}{\end{trivlist}}
\newenvironment{exercise}[2][Exercise]{\begin{trivlist}
\item[\hskip \labelsep {\bfseries #1}\hskip \labelsep {\bfseries #2.}]}{\end{trivlist}}
\newenvironment{problem}[2][Problem]{\begin{trivlist}
\item[\hskip \labelsep {\bfseries #1}\hskip \labelsep {\bfseries #2.}]}{\end{trivlist}}
\newenvironment{question}[2][Question]{\begin{trivlist}
\item[\hskip \labelsep {\bfseries #1}\hskip \labelsep {\bfseries #2.}]}{\end{trivlist}}
\newenvironment{corollary}[2][Corollary]{\begin{trivlist}
\item[\hskip \labelsep {\bfseries #1}\hskip \labelsep {\bfseries #2.}]}{\end{trivlist}}
\newcommand{\textfrac}[2]{\dfrac{\text{#1}}{\text{#2}}}
\newcommand{\floor}[1]{\left\lfloor #1 \right\rfloor}

\newenvironment{amatrix}[1]{%
  \left(\begin{array}{@{}*{#1}{c}|c@{}}
}{%
  \end{array}\right)
}

\DeclareMathOperator*{\E}{\mathbb{E}}


\begin{document}

\title{Dynamical Systems: Homework 3}

\author{Chris Hayduk}
\date{October 8, 2020}

\maketitle

\begin{problem}{1}
\end{problem}

Recall that $\Sigma_b^+$ is a one-sided shift map. That is,
\begin{align*}
\Sigma_b^+ = \{x = (x_k)_{k=0}^{\infty} : x_k \in \{0, \cdots b-1\}\}
\end{align*}

The shift map $\sigma: \Sigma_b^+ \to \Sigma_b^+$ is defined as follows:
\begin{align*}
\sigma((x_k)_{k=0}^{\infty}) = (x_{k+1})_{k = 0}^{\infty}
\end{align*}

That is, we are essentially cutting off the first element of the sequence.\\

Lastly, we have that,
\begin{align*}
d(x, y) = d_{\theta}(x, y) = \begin{cases}
      0 & \text{if} \; x = y \\
      \theta^{\min \{k: x_k \neq y_k\}} & \text{if} \; x \neq y
    \end{cases}
\end{align*}

where $0 < \theta < 1$.\\

We will start now by proving that $\sigma$ has sensitive dependence.\\

Now fix $x = (x_0x_1x_2\cdots) \in \Sigma_b^+$ and let $\delta > 0$. Fix $k \in \mathbb{N}$ such that it is the smallest natural number with $\theta^k < \delta$. Now define $y$ such that $y_i = x_i$ for $0 \leq i \leq k-1$ and such that, for every index $i > k$, we have $y_i \in \{0, \cdots, b-1\}$ and $y_i \neq x_i$. Clearly $y \in \Sigma_b^+$ and we have,
\begin{align*}
y = (x_0x_1x_2\cdots x_{k-1}y_{k}y_{k+1} \cdots)\\
x = (x_0x_1x_2\cdots x_{k-1}x_kx_{k+1} \cdots)
\end{align*}

This yields,
\begin{align*}
d(x, y) = \theta^k < \delta
\end{align*}

Now apply the shift map $k$ times:
\begin{align*}
\sigma^{k}(x) &= (x_kx_{k+1}x_{k+2}\cdots) = (\sigma^k(x)_0\sigma^k(x)_1\cdots)\\
\sigma^{k}(y) &= (x_ky_{k+1}y_{k+2}\cdots) = (\sigma^k(y)_0\sigma^k(y)_1\cdots)
\end{align*}

Then we have,
\begin{align*}
d(\sigma^{k}(x), \sigma^{k}(y)) = \theta^1
\end{align*}

If we fix $\epsilon = \theta^2$, then we have,
\begin{align*}
d(\sigma^{k}(x), \sigma^{k}(y)) = \theta^1 > \epsilon
\end{align*}

Since $x$ and $\delta$ were arbitrary, this holds for every $x \in \Sigma_b^+$ and every $\delta > 0$. Hence, $\sigma$ has sensitive dependence.\\

Now we will prove that $\text{Per}(\sigma)$ is dense in $\Sigma_b^+$.\\

For a point $x \in \Sigma_b^+$ to be periodic, it must be the repetition for some block of symbols in $\{1, \cdots, b-1\}$. That is, there is some $n \in \mathbb{N}$ such that $x_{[0, n-1]} = x_{[n, 2n-1]} = x_{[2n, 3n-1]} = \cdots$.\\

Now fix $y \in \Sigma_b^+$ and fix $\epsilon > 0$. Then consider the set,
\begin{align*}
B_{\epsilon}(y) = \{x \in \Sigma_b^+: d(x, y)\} < \epsilon
\end{align*}

So for every $x \neq y \in B_{\epsilon}(y)$, we have $d(x, y) = \theta^{\min \{k: x_k \neq y_k\}} < \epsilon$\\

Now let $k \in \mathbb{N}$ be the smallest natural number such that $\theta^k < \epsilon$. Then define the block $y_{[0, k-1]}$ using elements of $y$. Next, define $x_k = y_k + 1 \mod b$. Let $x_y = (y_{[0, k-1]}x_ky_{[0, k-1]}x_ky_{[0, k-1]}x_k\cdots) = (y_0y_1\cdots y_{k-1}x_ky_0y_1\cdots y_{k-1}x_k\cdots)$. Clearly we have,
\begin{align*}
d(x_y, y) &= \theta^k < \epsilon
\end{align*}

Moreover, we have that $x_{[0, k]} = x_{[k+1, 2k+1]} = x_{[2k+2, 3k+2]} = \cdots$. So we have that,
\begin{align*}
\sigma^{k+1}(x) = x
\end{align*}

Hence, $x$ is a periodic point with period $k+1$.\\

Since $\epsilon > 0$ and $y \in \Sigma_b^+$ were arbitrary, this holds for any $y \in \Sigma_b^+$ with any choice of $\epsilon > 0$. Hence, we have Per($\sigma$) is dense in $\Sigma_b^+$.\\

Lastly we need to show that $\sigma$ is topologically transitive. That is, $\exists x \in \Sigma_b^+$ such that the orbit $O(x) = \{\sigma^n(x) : n \in \mathbb{N}_0\}$ of $x$ is dense in $\Sigma_b^+$.\\

Define $x$ to be the sequence of all valid $n$ blocks in $\Sigma_b^+$. That is, for every valid block of symbols in $\Sigma_b^+$, such as $z_1z_2z_3\cdots z_n$, this block is contained somewhere in $x$. The number of valid blocks is countable because for each block size $n$, there are a finite number of combinations. Hence, $x$ is essentially a countable union of finite sets and thus is countable. As a result, $x$ is a valid element of $\Sigma_b^+$.\\

Now fix $y \in \Sigma_b^+$. Let $\epsilon > 0$ and define $k \in \mathbb{N}$ such that it is the smallest natural number such that $\theta^k < \epsilon$.\\

Now consider the first $k$ elements of $y$. That is,
\begin{align*}
y_{[0, k-1]} = y_0y_1\cdots y_{k-1}
\end{align*}

We have that $y_{[0, k-1]}$ is a $k$-block in $\Sigma_b^+$ and that $x$ contains every possible block at some point in the sequence. So we know we can apply $\sigma$ some number $m$ times which yields,
\begin{align*}
\sigma^m(x)_{[0, k-1]} = y_{[0, k-1]}
\end{align*}

Then we have that,
\begin{align*}
d(\sigma^m(x), y) = \theta^k < \epsilon
\end{align*}

So we have $\sigma^m(x) \in B_{\epsilon}(y)$. Since $y \in \Sigma_b^+$ and $\epsilon > 0$ were arbitrary, this holds for any $y \in \Sigma_b^+$ with any choice $\epsilon > 0$. Hence, we have that $O(x)$ is dense in $\Sigma_b^+$.\\

Hence, we have that $\sigma$ satisfies all three properties and hence $\sigma$ is chaotic.

\begin{problem}{2}
\end{problem}

Recall that $f$ and $g$ being topologically conjugate means there exists a homeomorphism $h: A \to B$ such that,
\begin{align*}
h \circ f \circ h^{-1} = g
\end{align*}

Or equivalently,
\begin{align*}
h \circ f = g \circ h
\end{align*}

In addition, $h$ as a homeomorphism means that it is bijective, continuous, and its inverse $h^{-1}$ is continuous.\\

Now suppose $f$ is chaotic. We know that $f$ is topologically transitive. That is, $\exists x \in A$ such that $O(x)$ is dense in $A$.\\

Fix $y \in B$. Let $\epsilon > 0$ and consider the open set $B_{\epsilon}(y)$. We have that $h^{-1}$ is continuous. Take $h^{-1}(y_1) = x$

\begin{problem}{3}
\end{problem}

\begin{problem}{4}
\end{problem}

\end{document}