\documentclass[12pt]{article}
 
\usepackage[margin=1in]{geometry}
\usepackage{amsmath,amsthm,amssymb, mathtools}
\usepackage[T1]{fontenc}
\usepackage{lmodern}
\usepackage{fixltx2e}
\usepackage[shortlabels]{enumitem}
\usepackage{mathrsfs}
\usepackage{kbordermatrix}

\usepackage{graphicx}

\renewcommand{\kbldelim}{(}% Left delimiter
\renewcommand{\kbrdelim}{)}% Right delimiter
 
\newcommand{\N}{\mathbb{N}}
\newcommand{\R}{\mathbb{R}}
\newcommand{\Z}{\mathbb{Z}}
\newcommand{\Q}{\mathbb{Q}}
 
\newenvironment{theorem}[2][Theorem]{\begin{trivlist}
\item[\hskip \labelsep {\bfseries #1}\hskip \labelsep {\bfseries #2.}]}{\end{trivlist}}
\newenvironment{lemma}[2][Lemma]{\begin{trivlist}
\item[\hskip \labelsep {\bfseries #1}\hskip \labelsep {\bfseries #2.}]}{\end{trivlist}}
\newenvironment{exercise}[2][Exercise]{\begin{trivlist}
\item[\hskip \labelsep {\bfseries #1}\hskip \labelsep {\bfseries #2.}]}{\end{trivlist}}
\newenvironment{problem}[2][Problem]{\begin{trivlist}
\item[\hskip \labelsep {\bfseries #1}\hskip \labelsep {\bfseries #2.}]}{\end{trivlist}}
\newenvironment{question}[2][Question]{\begin{trivlist}
\item[\hskip \labelsep {\bfseries #1}\hskip \labelsep {\bfseries #2.}]}{\end{trivlist}}
\newenvironment{corollary}[2][Corollary]{\begin{trivlist}
\item[\hskip \labelsep {\bfseries #1}\hskip \labelsep {\bfseries #2.}]}{\end{trivlist}}
\newcommand{\textfrac}[2]{\dfrac{\text{#1}}{\text{#2}}}
\newcommand{\floor}[1]{\left\lfloor #1 \right\rfloor}

\newenvironment{amatrix}[1]{%
  \left(\begin{array}{@{}*{#1}{c}|c@{}}
}{%
  \end{array}\right)
}

\DeclareMathOperator*{\E}{\mathbb{E}}

\newcommand{\Mod}[1]{\ (\mathrm{mod}\ #1)}

\begin{document}

\title{Dynamical Systems: Midterm}

\author{Chris Hayduk}
\date{November 1, 2020}

\maketitle

\begin{problem}{1}
\end{problem}

\begin{enumerate}[label=(\alph*)]

\item Devaney's definition of a chaotic function is as follows,\\

\textbf{Definition} Let $V$ be a set. $f: V \to V$ is said to be chaotic on $V$ if
\begin{enumerate}

\item $f$ has \textit{sensitive dependence on initial conditions}. That is, $\exists \delta > 0$ such that for any $x \in V$ and all $\epsilon > 0$, there exists $y \in B(x, \epsilon)$ and $n \geq 0$ such that $|f^n(x) - f^n(y)| > \delta$ 

\item $f$ is \textit{topologically transitive}. That is, $\exists x \in V$ such that $O(x)$ is dense in $V$

\item periodic points of $f$ (i.e. Per$(f)$) are dense in $V$

\end{enumerate}

%%%%%%%%%%%%%%%%%%%%%%%%%%%%%%%%%%%%%%%%%%%%%%%%%%%%%

\item Note that on HW3, Q4 we discussed the angle doubling map, $g: S^1 \to S^1$ given by $z \mapsto z^2$. This was equivalent to $\tilde{g}: S^1 \to S^1$, $\tilde{g}(\alpha) = e^{2\pi i (2\alpha)} = e^{4\pi i \alpha}$.\\

Let us now turn our attention to this function: $f: S^1 \to S^1$ given by $z \mapsto z^5$. This is equivalent to $\tilde{f}: S^1 \to S^1$, $\tilde{f}(\alpha) = e^{2\pi i (5\alpha)} = e^{10\pi i \alpha}$.\\

We proved in HW3, Q4 that $\tilde{g}$ (and hence $g$) was chaotic. Moreover, we proved in HW3, Q3 if two continuous maps defined on metric spaces are topologically conjugate, then one being chaotic implies that the other is chaotic. Since $\tilde{g}$ and $\tilde{f}$ are both continuous and defined on metric spaces, all we must show is that $\tilde{g}$ is topologically conjugate to $\tilde{f}$ in order to show that $f$ is chaotic.\\

Note that $h(x) = e^{2/5}x$ is continuous since it is a constant multiplied by $x$, where $x$ is in $S^1$. In addition, we have that $h^{-1}(x) = e^{5/2}x$ is continuous for the same reason. Both are also well-defined on $S^1$. Moreover, note that 
\begin{align*}
h \circ \tilde{f} &= e^{2/5} \cdot e^{10\pi i \alpha}\\
&= e^{4\pi i \alpha}
\end{align*}

\end{enumerate}

%%%%%%%%%%%%%%%%%%%%%%%%%%%%%%%%%%%%%%%%%%%%%%%%%%%%%%%%%%%%%%%%%%%%%%%%
%%%%%%%%%%%%%%%%%%%%%%%%%%%%%%%%%%%%%%%%%%%%%%%%%%%%%%%%%%%%%%%%%%%%%%%%
%%%%%%%%%%%%%%%%%%%%%%%%%%%%%%%%%%%%%%%%%%%%%%%%%%%%%%%%%%%%%%%%%%%%%%%%
%%%%%%%%%%%%%%%%%%%%%%%%%%%%%%%%%%%%%%%%%%%%%%%%%%%%%%%%%%%%%%%%%%%%%%%%
\newpage
\begin{problem}{2}
\end{problem}

\begin{enumerate}[label=(\alph*)]

\item Let $\lambda = -e$. Then,
\begin{align*}
E_{\lambda}(x) &= \lambda e^x\\
&= -e \cdot e^x\\
&= -e^{x+1}
\end{align*}

Hence, we have,
\begin{align*}
E_{\lambda}(-1) &= -e^{-1 + 1}\\
&= -e^0\\
&= -1
\end{align*}

Now note that, by the chain rule, we have
\begin{align*}
E_{\lambda}'(x) &= -e^{x+1}\\
&= E_{\lambda}(x)
\end{align*}

Thus,
\begin{align*}
E_{\lambda}'(-1) &= E_{\lambda}(-1)\\
&= -1
\end{align*}

%%%%%%%%%%%%%%%%%%%%%%%%%%%%%%%%%%%%%%%%%%%%%%%%%%%%%

\item Let $\lambda > -e$. Then,
\begin{align*}
E_{\lambda}(x) > -e^{x+1}
\end{align*}

%%%%%%%%%%%%%%%%%%%%%%%%%%%%%%%%%%%%%%%%%%%%%%%%%%%%%

\item Let $\lambda < -e$. Then,
\begin{align*}
E_{\lambda}(x) < -e^{x+1}
\end{align*}

\end{enumerate}

%%%%%%%%%%%%%%%%%%%%%%%%%%%%%%%%%%%%%%%%%%%%%%%%%%%%%%%%%%%%%%%%%%%%%%%%
%%%%%%%%%%%%%%%%%%%%%%%%%%%%%%%%%%%%%%%%%%%%%%%%%%%%%%%%%%%%%%%%%%%%%%%%
%%%%%%%%%%%%%%%%%%%%%%%%%%%%%%%%%%%%%%%%%%%%%%%%%%%%%%%%%%%%%%%%%%%%%%%%
%%%%%%%%%%%%%%%%%%%%%%%%%%%%%%%%%%%%%%%%%%%%%%%%%%%%%%%%%%%%%%%%%%%%%%%%

\begin{problem}{3}
\end{problem}

\begin{enumerate}[label=(\alph*)]

\item 

%%%%%%%%%%%%%%%%%%%%%%%%%%%%%%%%%%%%%%%%%%%%%%%%%%%%%

\item

\end{enumerate}

%%%%%%%%%%%%%%%%%%%%%%%%%%%%%%%%%%%%%%%%%%%%%%%%%%%%%%%%%%%%%%%%%%%%%%%%
%%%%%%%%%%%%%%%%%%%%%%%%%%%%%%%%%%%%%%%%%%%%%%%%%%%%%%%%%%%%%%%%%%%%%%%%
%%%%%%%%%%%%%%%%%%%%%%%%%%%%%%%%%%%%%%%%%%%%%%%%%%%%%%%%%%%%%%%%%%%%%%%%
%%%%%%%%%%%%%%%%%%%%%%%%%%%%%%%%%%%%%%%%%%%%%%%%%%%%%%%%%%%%%%%%%%%%%%%%

\begin{problem}{4}
\end{problem}

\begin{enumerate}[label=(\alph*)]

\item Idea: assign to each dynamical system a positive real number $h_{\text{top}}(f)$, where $f: X \to X$ is continuous and $(X, d)$ is a compact metric space such that
\begin{enumerate}[label=\roman*)]
\item if $f$ and $g$ are topologically conjugate then $h_{\text{top}}(f) = h_{\text{top}}(g)$
\item if $0 < h_{\text{top}}(f) < h_{\text{top}}(g)$, then $g$ is more chaotic than $f$
\end{enumerate}

Now let $f: X \to X$. Entropy will be the exponential growth rate of the number of distinct orbits. We say $\epsilon > 0$ is the ``margin of measurement''.\\

Iterate $n-$times, $x$ and $y$:
\begin{align*}
x \to f(x) \to f^2(x) &\to \cdots \to f^{n-1}(x)\\
y \to f(y) \to f^2(y) &\to \cdots \to f^{n-1}(y)
\end{align*}

Now define,
\begin{align*}
d_n(x,y) &= \max_{k = 0, \cdots, n-1} d(f^k(x), f^k(y))
\end{align*}

That is, $d_n(x, y)$ denotes the maximum distance between $x$ and $y$ along the orbit segment of length $n$. $d_n$ is also called the Bowen metric.\\

Now let $F \subset X$. Also let $\epsilon > 0$ and $n \in \mathbb{N}$ be fixed. F is called $(n, \epsilon)-$separated if $\forall x, y \in F$ with $x \neq y$, we have $d_n(x, y) \geq \epsilon$.\\

Assume $F_n(\epsilon)$ is a maximal $(n, \epsilon)-$separated set. That is
\begin{enumerate}[label = \roman*)]
\item $F_n(\epsilon)$ is $(n, \epsilon)$ separated
\item if $y \in X \setminus F_n(\epsilon)$, then $I_n(\epsilon) \cup \{y\}$ is not $(n, \epsilon)-$separated
\end{enumerate}

\textbf{Remark:} Since $X$ is compact, $F_n(\epsilon)$ is finite\\

Now define $h_{\text{top}}(f)$ as,
\begin{align*}
h_{\text{top}}(f) &= \lim_{\epsilon \to 0} \overline{\lim_{n \to \infty}} \frac{1}{n} \log \left(\text{cardinality of} \ F_n{\epsilon}\right)
\end{align*}
%%%%%%%%%%%%%%%%%%%%%%%%%%%%%%%%%%%%%%%%%%%%%%%%%%%%%

\item Let $r > 0$. Let us define,
\begin{align*}
A =  \kbordermatrix{
    & 0 & 1  \\
    0 & 1 & 1\\
    1 & 1 & 1
  }
\end{align*}

Then we have that $\Sigma^+_A = \Sigma^+_2$. Hence, let us define $f = \sigma_A$ as our continuous map from $\Sigma^+_2 \to \Sigma_2^+$. Thus, we have that,
\begin{align*}
h_{\text{top}}(f) &= h_{\text{top}}(\sigma_A)\\
&= \log \rho
\end{align*}

where $\rho$ is the largest eigenvalue of $A$. Note that the two eigenvalues of $A$ are $2$ and $0$. Hence, we have $\rho = 2$ and,
\begin{align*}
h_{\text{top}}(f) &= \log 2\\
&\approx 0.693
\end{align*}

Now let us use the fact that $2h_{\text{top}}(f) = h_{\text{top}}(f^2)$. Define $g = f^2$. Then we have,
\begin{align*}
2h_{\text{top}}(g) &= h_{\text{top}}(g^2)\\
\end{align*}

But this implies,
\begin{align*}
2h_{\text{top}}(f^2) &= 2[2h_{\text{top}}(f)]\\
&= 4h_{\text{top}}(f)
\end{align*}

Proceeding by induction, we see that $2^kh_{\text{top}}(f) = h_{\text{top}}(f^{2k})$ for $k \in \mathbb{N}$. Thus, in our case, 
\begin{align*}
 h_{\text{top}}(f^{2k}) = 2^k \cdot \log 2
\end{align*}

Now we need $k$ such that,
\begin{align*}
r < 2^k \cdot \log 2
\end{align*}

which implies,
\begin{align*}
&\log_2 \left(\frac{r}{\log 2}\right) < k\\
\end{align*}

So if we choose $k \in \mathbb{N}$ such that $k > \log_2 \left(\frac{r}{\log 2}\right)$, then we have that,
\begin{align*}
h_{\text{top}}(f^{2k}) > r
\end{align*}

Now let $g = f^{2k} = \sigma_A^{2k}$. We have that $g: \Sigma_2^+ \to \Sigma_2^+$, $g$ continuous, and
\begin{align*}
h_{\text{top}}(g) > r
\end{align*}
%%%%%%%%%%%%%%%%%%%%%%%%%%%%%%%%%%%%%%%%%%%%%%%%%%%%%

\item Yes, it is possible for a continuous $f: \mathbb{R} \to \mathbb{R}$ to have precisely one periodic orbit. For example, consider $f(x) = x/2$. Clearly this function is continuous on all of $\mathbb{R}$ and $f(0) = 0$, so $0$ has period 1. But observe that, for any $x \neq 0$, we have,
\begin{align*}
\cdots < f^3(x) = x/8 < f^2(x) = x/4 < f(x) = x/2 < x
\end{align*}

Thus, we see that the sequence is $(f^n(x))_{n=0}^{\infty}$ is actually equivalent to 
\begin{align*}
(x/2^n)_{n=0}^{\infty}
\end{align*}

This sequence is strictly decreasing. Hence, for any choice of $x$, there will never be any point $n_1 \in \mathbb{N}$ with $n_1 > 0$ such that $x_{n_1} = x$ since we must have that $x_{n_1} < x$. Since $x \neq 0$ was arbitrary in $\mathbb{R}$, we have shown that there are no periodic points in $\mathbb{R} \setminus \{0\}$. Thus, $0$ is the only periodic point of $f$ and has period $1$.
\end{enumerate}

\end{document}